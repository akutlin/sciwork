\documentclass{article}
\usepackage{indentfirst}
\usepackage{graphicx}
\usepackage{caption}

\topmargin=-20mm
\textheight=230mm
\oddsidemargin=-10mm
\textwidth=180mm

\begin{document}

\centerline{{\bf Ritz method and arbitrary order approximation for the Stokes constants}}
\vspace{5mm}
Lets discuss an eigenvalue problem for $Q(x)=E-x^4$. After all
transformations associated with the symmetry relations the problem 
of spectre's calculation can be reduced to solving of an equation
\begin{eqnarray}
f(Ee^{2i\pi/3})+f(Ee^{-2i\pi/3})=0,   \label{fSpctr}
\end{eqnarray}
where function $f(E)$ is connected with an original Stokes constant via $S(Arg(x)=0,E)=ie^{-CE^{3/4}}f(E)$. The exponent $W=CE^{3/4}$ is a part of a phase integral - see our article, chapter 4. Function $f(E)$ can be found from a functional equation
\begin{eqnarray}
f(E)+f(Ee^{2i\pi/3})+f(Ee^{-2i\pi/3})=f(E)f(Ee^{2i\pi/3})f(Ee^{-2i\pi/3})   \label{fEq}
\end{eqnarray}
using the following conditions:\\
$(a) f(E^{*}) = f^{*}(E)$\\
$(b) f(E) = f(Ee^{2i\pi})$\\
$(c) f(0) = \sqrt{3}$\\
$(d) f(E) \sim e^{CE^{3/4}}, E \rightarrow \infty$.\\
The first three conditions are faithfully true and the last one is true only if an approximation of isolated singularities is true (why is it so? what conditions?).

This problem is too difficult for being solved exactly - I still do not know how to do it. But it can be solved approximately using one more condition - the behaviour of the Stokes constants for the large separation between singularities. How I have already said I checked four potentials which allow an exact determination of Stokes constants and found that $|i-S(W)| \sim const/W$. Assuming this is a general fact we can write fifth condition for $f(E)$:
\begin{eqnarray}
1-f(E)e^{-CE^{3/4}} \sim const/W. \label{fifth}
\end{eqnarray}

All further operations are based on the Ritz's idea: we define a trial function which depends on some parameters and for any values of the parameters satisfy the conditions written above with one exception: since some of our conditions are asymptotic ((d) and Eq.\ref{fifth}) and we want to find a good approximation only in some wedge of the complex plane, we can forget about condition $(b)$ because it is a global condition and can be violated for an asymptotic series solution. While we cannot solve Eq.\ref{fEq}, we can still use it as a measure of the rightness of our approximate solution. For more clearness let's change our variable and function the following way:
\begin{eqnarray}
g(W)=1/f(E(W))=1/f\left(\frac{W^{4/3}}{C^{4/3}}\right). \label{changes}
\end{eqnarray}
In the new variables our problem looks like
\begin{eqnarray}
g(W)g(iW)+g(W)g(-iW)+g(iW)g(-iW)=1. \label{gEq}
\end{eqnarray}

Now lets define an integral measure of the correctness of our solution:
\begin{eqnarray}
M[f(x)]=\int_0^\infty |1-f(x)g(ix)-f(x)f(-ix)-f(ix)f(-ix)|^2 dx. \label{IM}
\end{eqnarray}
$M[f(x)]$ is obviously zero for an exact solution and positive for other functions. Now we can choose a trial function and reduce our problem to a search of minimum of the multidimensional function $m(parameters)=M[f(x,paramerets)]$ - a simple optimisation problem, much simpler than the initial one.

According to our conditions for $f(E)$, $g_{trial}(W)$ can be chosen as
\begin{eqnarray}
g_{trial}^{(N)}(W) = 
\frac{1+W^{N} \left(c_0 + \sum_{n=1}^{n=N-1}{a_n/W^n}\right)}
{\sqrt{3}+W^{N} \left(c_0 + \sum_{n=1}^{n=N-1}{b_n/W^n}\right)} e^{-W}. \label{trial}
\end{eqnarray} 
Maybe it can be chosen differently, but this is the simplest general form that I came up with. Calculating $m[g_{trial}^{(N)}(W)]$
and minimizing it we find optimal values for $c_0, a_n$ and $b_n$. In the simplest case of $N=1$ we find
\begin{eqnarray}
g_{opt}^{(1)}(W) = \frac{1+2 W}{\sqrt{3}+2 W} e^{-W}. \label{opt1}
\end{eqnarray}
Different orders of approximation connected via $c_0=0$. The results up to the third order are summarized in the table below. Approximation's orders differ by $RN$ energy index.
\begin{eqnarray}
\left( \begin{array}{c|c|c|c|c|c|c}
\hline
n & E_{exact}  & E_{wkb} & E_{PI} & E_{R1} & E_{R2} & E_{R3} \\
\hline
0 & 1.0604 & 0.8671&   1.0246 & 1.0034 & 1.0475 & 1.0627\\
1 & 3.7964 & 3.7519 &   3.7424 & 3.8320 & 3.8123 & 3.7999\\
2 & 7.45567 & 7.4139 &  7.4144 & 7.4722 & 7.4515 & 7.4543\\
3 & 11.6374 & 11.6114 &  11.6114 & 11.6583 & 11.6402 & 11.6456\\
4 & 16.2618 & 16.2335 &  16.2335 & 16.2733 & 16.2574 & 16.2632
\nonumber
\end{array}\right)
\end{eqnarray}
It can be easily seen that, in contrast to $E_{PI}$, $E_{RN}$ gives a visible correction to WKB value even for high energy levels.

\end{document}