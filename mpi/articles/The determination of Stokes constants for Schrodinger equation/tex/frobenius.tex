\documentclass[aip,jmp,reprint]{revtex4-1}
\usepackage{graphicx}
\usepackage{amsmath}
\usepackage{bm}

\def\rmi{\mathrm{i}}
\def\rme{\mathrm{e}}
\def\rmd{\mathrm{d}}

\def\R{\widehat{\mathcal{R}}}
\def\psii{\bm\psi}
\def\T{\mathrm{T}}
\def\Tr{\mathrm{Tr}}
\def\mytextwidth{0.47\textwidth}

%\def\S{\widehat{S}}
%\def\W{\widehat{W}}
%\def\C{\widehat{C}}
%\def\f{\hat{f}}
%\def\g{\hat{g}}
%\def\h{\hat{h}}
%\def\L{\widehat{L}}
%\def\lmbd{\bm{\lambda}}
%\def\unity{\hat{\it 1}}
%\def\Re{\mathrm{Re}}
%\def\Im{\mathrm{Im}}
\def\w{\omega}

\newcommand\phsintgrnd[1][z]{q(#1)}
\newcommand\predexp[1][z]{q(#1)^{-1/2}}
\newcommand\phsintgrl[3][z]{\int_{#2}^{#3} \phsintgrnd[#1] \rmd #1}

\begin{document}

\title{On a new nontrivial exact relation for the Stokes constants for a solution of 
a linear second-order ODE with non-analytic coefficients}
\author{Anton Kutlin}
\email{anton.kutlin@gmail.com}
\affiliation{Institute of Applied Physics of Russian Academy of Sciences, 
46 Ulyanov str., 603950 Nizhny Novgorod, Russia}

\date{\today}

\begin{abstract}
We consider an ordinary linear second-order differential equation with non-analytical coefficients.
We propose a universal technique based on the Frobenius method which allows to obtain new nontrivial 
exact relation between Stokes constants providing the differential equation has one regular singular point.
The well-known Budden problem is solved using this technique as an illustration of the possible usage
of the technique. We also propose a way to write approximate relations between Stokes constants 
if the equation has multiple regular singular points located far away from each other.
\end{abstract}

\pacs{02.30.Hq,02.30.Mv}
\keywords{Stokes phenomenon, phase-integral method, Frobenius method}

\maketitle

\section{introduction \label{sec:intro}}
Consider an arbitrary ordinary linear differential equation written in the form of 
a stationary one-dimensional Schr\"odinger equation:
\begin{eqnarray}
\frac{d^2y}{dz^2} + Q(z)y = 0,   \label{eq:gen}
\end{eqnarray}
where $Q(z)$ will be referred to as a potential. Its approximate local solution could be obtained
with use of the phase integral approximation. Provided that 
\begin{eqnarray}
\varepsilon = q^{-3/2} \rmd^2 q^{-1/2}/\rmd z^2  + (Q - q^2)/q \ll 1,   \label{eq:cond}
\end{eqnarray}
in the most general form a solution of \eqref{eq:gen} can be approximated by
\begin{eqnarray}
y = c_+y_+ + c_-y_-, \\ \label{eq:gensol}
y_\pm = \predexp \exp [\pm \rmi \w(z)], \quad \w(z)=\phsintgrl[\xi]{(z_0)}{z},   \label{eq:phsint}
\end{eqnarray}
where the explicit form of $q(z)$ depends on the order of the approximation.
The function $\w(z)$ is the phase integral, and therefore we call $\phsintgrnd[\xi]$ 
the phase integrand. A meaning of the brackets in the lower limit of integration is a 
bit tricky; such a notation was introduced by Fr\"omans \cite{frpaper} to make
phase integral look similar for all orders of approximation. In the first order this integral 
is just a usual integral from $z_0$ to $z$. 

Any phase-integral type solution \eqref{eq:phsint} is a local, not global solution of \eqref{eq:gen}. 
The method of phase integrals allows the construction of a globally defined 
asymptotic expression for the solution of a desired linear ordinary differential 
equation. The method was first proposed by A.Zwaan in his dissertation in 1929 \cite{zwaan}. 
He suggested allowing the independent variable in the differential equation to take 
complex values and to study a behaviour of the asymptotic solution far away from any 
singularities. According to Stokes\cite{stokes}, for any given exact solution 
of \eqref{eq:gen} coefficients $c_+$ and $c_-$ in the approximate solution \eqref{eq:gensol} 
differ from one domain of the complex plane to another 
(Stokes phenomenon \cite{stokes,rwbook,heading,frbook}). Such abrupt 
changes happen on the so-called Stokes lines and have a form of a single-parameter 
linear transform\cite{heading}. The parameter associated with a particular Stokes line 
is called the Stokes constant. Knowing all Stokes constants associated with a particular 
potential gives the ability\cite{heading,rwbook} to obtain a globally defined approximate solution 
of \eqref{eq:gen}. A standard technique\cite{frpaper} used to obtain
equations for the Stokes constants is based on the assumption of a single-valuedness of
an exact general solution. This assumption fails for equations with regular singular points, thus
the only way to solve such equations is to use some approximation for the Stokes constants
(e.g. the approximation of isolated singularities). In the present paper we provide a universal
technique to write equations for the Stokes constants. In case of a differential equation with
a single regular singular point our equation is exact.

The paper is organized as follows.
In section \ref{sec:frob} we derive an exact equation for the Stokes constants in case of the differential
equation with the only one regular singular point. 
In section \ref{sec:budden} our technique is used to solve the well-known Budden problem. Also this
section contains a comparison of our result with an exact one and the result obtained with use
of the approximation of isolated singularities.
%In section \ref{sec:approx} the possibility of writing approximate equation in case of multiple 
%regular singular points discussed.
And, finally, in section \ref{sec:con} are the conclusions. 

\section{The exact equation for the Stokes constants in case of a single regular singular point \label{sec:frob}}
Consider equation \eqref{eq:gen} with a single regular singular point. Without loss of generality
we can place the singularity into the origin of the complex plane. A general solution of such equation 
can be written in a form of Frobenius series:
\begin{eqnarray}
y(z) = A y_1(z)+B y_2(z), \\
y_1(z) = z^{f_1}\sum_{n=0}^{\infty}{a_n z^n}, \\
y_2(z) = z^{f_2}\sum_{n=0}^{\infty}{b_n z^n} + K \ln(z) y_1(z),    \label{eq:fgensol}
\end{eqnarray}
where $A$ and $B$ are arbitrary constants depending on particular boundary/initial conditions,  
$f_{1,2}$, $a_n$, $b_n$ and $K$ are coefficients determined by direct substitution of this form into the 
differential equation, and $a_0=b_0=1$ by convention. Particularly, this substitution gives
an indicial equation for the Frobenius indexes $f_1$ and $f_2$:
\begin{eqnarray}
f(f-1)+Q_{-2}=0,   \label{eq:indicial}
\end{eqnarray}
where $Q_{-2}$ is a coefficient before the inverse square of the complex variable $z$ in the Laurent series
of the potential around the pole. Since $z=0$ is the only singular point of the potential, the Taylor 
series from the general solution \eqref{eq:fgensol} have infinite radii of convergence and the
Frobenius form of the solution is valid for any point of the entire complex plane.

Now consider a solution \eqref{eq:fgensol} with $A=1$ and $B=0$, i.e. $y(z)=y_1(z)$. 
As it can be seen from a substitution \mbox{$z \rightarrow z \rme^{2 \rmi \pi}$}, 
such solution is an eigenfunction of the operator of
$2\pi-$rotation about the origin with the eigenvalue $\lambda = \rme^{2 \rmi \pi f_1}$.  
Suppose that for some values of $z$ in the vicinity of complex infinity 
the solution can be written asymptotically in the form \eqref{eq:gensol} as
\begin{eqnarray}
y_1(z) \sim a_+y_+(z) + a_-y_-(z).
\end{eqnarray}
Now, using Heading`s rules of analytical continuation and a single-valuedness of squared phase integrand,
we can write that
\begin{eqnarray}
y_1(z \rme^{2 \rmi \pi}) \sim b_+y_+(z) + b_-y_-(z).
\end{eqnarray}
The coefficients $b_{\pm}$ are known in terms of the Stokes constants, phase integrals and 
the coefficients $a_{\pm}$.
Taking into consideration a linearity of equation \eqref{eq:gen}, the connection between $a_{\pm}$
and $b_{\pm}$ can be written in a simple matrix form as
\begin{eqnarray}
\psii_b = \R \psii_a,
\end{eqnarray}
where $\psii_a = [{a_+,a_-}]^{\T}$, $\psii_b = [{b_+,b_-}]^{\T}$ and 'T' denotes the transpose operation.
The matrix $\R$ represents the $2\pi-$rotation operator; it is expressed in terms of
the Stokes constants and corresponding phase integrals. According to the reasoning above we can conclude
that we know exactly one of the matrix`s eigenvalues---the eigenvalue is equal to $\rme^{2 \rmi \pi f_1}$.
Actually, we know another eigenvalue too. As it can be inferred from, for example, the matrix formulation 
of the Heading`s rules, $\det\R=1$ and the second eigenvalue is equal to $\rme^{-2 \rmi \pi f_1}$.

Now we can finally write the desired exact equation for the Stokes constants:
\begin{eqnarray}
\Tr\R = \rme^{2 \rmi \pi f_1} + \rme^{-2 \rmi \pi f_1}.
\label{eq:main}
\end{eqnarray}
This equation reflects an actual brunching structure of the general solution of equation \eqref{eq:gen}.

It is worth mentioning that $K$ from equation \eqref{eq:fgensol} may or may not be equal to zero. In the
former case one can notice that we can infer the second eigenvalue of $\R$ right from the
Frobenius form solution; it is equal to $\rme^{2 \rmi \pi f_2}$. Then one can propose to write two
separate equations using two different eigenvalues instead of a single equation \eqref{eq:main}.
Actually, these two equations will not be independent; particularly, it can be seen right from the
indicial equation \eqref{eq:indicial}. Indeed, an explicit form of these two eigenvalues
can be written as
\begin{eqnarray}
\lambda_{1,2} = - \exp(\pm \rmi \sqrt{1 - 4 Q_{-2}}),
\end{eqnarray}
from where follows $\det\R=1$ and all the reasoning that led to equation \eqref{eq:main}.

\section{Example: the Budden problem \label{sec:budden}}
Consider a differential equation described the standard problem of penetration and 
resonant absorption of an electromagnetic wave, analyzed by Budden\cite{white-chen,budden}
\begin{eqnarray}
\frac{\rmd^2 y(z)}{\rmd z^2} + (1+c/z)y = 0.  
\label{eq:budden}
\end{eqnarray}
The Budden problem is a problem of determination of the absorption in equation \eqref{eq:budden}. 
It can be solved exactly with use of an integral representation of the solution \cite{rwbook}.
Here we will examine this problem using the phase integral method and the technique presented above. 

The best-known type of the phase-integral approximation is the WKBJ (and often simply WKB) approximation, 
so named after Wentzel, Kramers, Brillouin, and Jeffreys \cite{wkb1,wkb2,wkb3,wkbj}. 
It takes the form \eqref{eq:phsint} with $\phsintgrnd = \sqrt{Q(z)}$. Since it is the simplest one,
we will use it all throughout this section. According to equation \eqref{eq:phsint}, 
equation \eqref{eq:budden} has $y_+ \propto e^{iz}$ and $y_- \propto e^{-iz}$ as its WKBJ asymptotic. 

Boundary conditions of equation \eqref{eq:budden} corresponding to the Budden problem can be
formulated as a presence of incident wave from the large negative $z$ and absence of such 
a wave from the large positive $z$. In terms of WKBJ asymptotics the conditions take the form
\begin{eqnarray}
y(z) \sim y_+(z), \quad z \rightarrow +\infty.  
\label{eq:bbound}
\end{eqnarray}
We define the reflection (transmission) coefficient $R$ ($T$) as
a ratio of the complex amplitudes of the reflected (transmitted) and incident waves. 
An absorption coefficient can be defined now as $A = 1 - |R|^2 - |T|^2$. Our aim in this
section is to find the absorption coefficient.

To calculate the absorption coefficient using the phase integral method, we must express 
reflection and transmission coefficients in terms of the Stokes constants. 
For this purpose we must analytically continue our solution from the large positive $z$,
where we know its asymptotic due to our boundary condition \eqref{eq:bbound}, to
the large negative $z$. The continuation must be implemented through the lower half of the
complex plane---it follows from the prescription of the absorption`s positiveness\cite{rwbook}.

\begin{figure}
\centering
\noindent
\includegraphics[width=\mytextwidth]{stuff/diagram.jpg}
\caption{Stokes diagram for the Budden problem; Stokes lines are bold and dashed}
\label{fig:diagram}
\end{figure} 

The analytical continuation could be done with use of the Heading`s rules. 
Following Heading\cite{heading}, define $(a,z) = Q^{-1/2}e^{i\int_a^z Q^{1/2} dz}$ 
and $[a,b] = e^{i\int_a^b Q^{1/2} dz}$. Let's place the cut between $z=0$ and $z=-c$ 
along the real axis as shown on Fig.\ref{fig:diagram}. Begin from the large positive $z$ 
with $y=(0,z)$ we obtain:
\begin{eqnarray}
1.\ (0,z)_d=[0,-c](-c,z)_s \nonumber\\
2.\ [0,-c](-c,z)_d \nonumber\\
3.\ [0,-c](-c,z)_d - s_-[0,-c](z,-c)_s \nonumber,
\end{eqnarray}
where $s_-$ is a Stokes constant corresponding to the lower half of the complex plane. 
All integrals here were evaluated below the cut and give $[0,-c]=e^{\frac{\pi c}{2}}$. As it
can be seen from the continuation and the definition of the scattering characteristics,
\begin{eqnarray}
R = -s_-, \quad T = e^{-\frac{\pi c}{2}}.
\label{eq:scattr}
\end{eqnarray}

To find $s_-$ and, thus, find $A$, we have to write the branching structure preserving 
equation \eqref{eq:main} obtained in the previous section. The $2\pi-$rotation operator $\R$
can be found from the analytical continuation of the general solution of \eqref{eq:budden}
around the origin. Starting with $y=a_+(0,z) + a_-(z,0)$ from large positive $z$, 
one can obtain that
\begin{eqnarray}
y(z \rme^{-2 \rmi \pi}) = b_+(0,z) + b_-(z,0),\\
b_+ = \rme^{c \pi} (1 + s_+s_-)a_+ - s_+a_-,\\
b_- = s_- a_+ + \rme^{-c \pi} a_-,
\label{eq:R}
\end{eqnarray}
%\begin{eqnarray}
%1.\ a_+(0,z)_d+a_-(z,0)_s=a_+[0,-c](-c,z)_s + a_-(z,-c)_d[-c,0] \nonumber\\
%2.\ a_+[0,-c](-c,z)_d + a_-[-c,0](z,-c)_s \nonumber\\
%3.\ a_+[0,-c](-c,z)_d + (a_-[-c,0] - R a_+[0,-c])(z,-c)_s \nonumber\\
%4.\ a_+[0,-c](-c,z)_s + (a_-[-c,0] - R a_+[0,-c])(z,-c)_d \nonumber\\
%5.\ (a_-[-c,0] - R a_+[0,-c])(z,-c)_d + (a_+[0,-c]-S(a_-[-c,0] - R a_+[0,-c]))(-c,z)_s \nonumber\\
%6.\ (a_-[-c,0] - R a_+[0,-c])(z,-c)_s + (a_+[0,-c]-S(a_-[-c,0] - R a_+[0,-c]))(-c,z)_d.
%\end{eqnarray}
where $s_+$ is the Stokes constant corresponding to the upper half of the complex plane.
As long as the Budden potential has only a first order pole, equation \eqref{eq:main}
with $f_\pm=0,1$ takes the form $\Tr\R=2$ or, using the explicit form of $\R$ from Eq.~\eqref{eq:R},
\begin{eqnarray}
s_+s_- = - (1-\rme^{- \pi c})^2.
\label{eq:fbudres}
\end{eqnarray}

This nontrivial exact equation couldn't have been obtained from any other considerations. But we
still have only one equation and two unknowns. To proceed further we have to find one another
relation between the Stokes constants; and it could be done with use of the symmetry 
of equation \eqref{eq:budden}. The Budden equation has a real potential on the real axis and
this fact allows to write\cite{symmetries} that
\begin{eqnarray}
s_+ = -s_-^*.
\label{eq:symmerty}
\end{eqnarray}
Now, considering equations \eqref{eq:scattr}, \eqref{eq:fbudres} and \eqref{eq:symmerty}, we can
finally write that
\begin{eqnarray}
A = e^{-\pi c}(1-e^{-\pi c}).
\label{eq:fabsorp}
\end{eqnarray}
The expression \eqref{eq:fabsorp} agrees with the analytical result\cite{rwbook}.

It could be useful to compare exact expression \eqref{eq:fabsorp} with an approximate result 
obtained with use of approximation of isolated singularities. According to the 
approximation\cite{rwbook}, reflection and transmission coefficients are
\begin{eqnarray}
R=-i\frac{2e^{\frac{\pi c}{2}}-e^{-\frac{\pi c}{2}}}{2e^{\frac{\pi c}{2}}+e^{-\frac{\pi c}{2}}}, \quad 
T=-i\frac{2}{2e^{\frac{\pi c}{2}}+e^{-\frac{\pi c}{2}}},
\end{eqnarray}
and for absorption
\begin{eqnarray}
A = \frac{4 e^{\pi c}}{(1+2e^{\pi c})^2}.
\end{eqnarray}

\begin{figure}
\centering
\noindent
\includegraphics[width=\mytextwidth]{stuff/comparison.jpg}
\caption{Absorption in the Budden potential}
\label{cmprsn}
\end{figure} 

Comparing these two expressions for absorption we can see that they behave similarly when $c\geq1$, 
but not when $c \sim 1$. The limit of $c\rightarrow 0$ is clearly wrong; in this limit $Q=1$
and there should be no reflection or absorption. The error is due to using the Stokes constants 
which do not preserve the global structure of the solution with a branch point. 

%\section{A case of an equation without an exact solution \label{UHR}}
%The example of the Budden problem is, certainly, a rare example of the problem which can be completely solved. In most cases we will not be able to solve an equation analogous to Eq.\ref{frobresult} exactly because of large amount of unknowns and lack of other equations. But since such an equation is highly non trivial, it can be used to elucidate unobvious features of a solution even in a such situations. To illustrate it, let`s study the following equation:
%\begin{eqnarray}
%\frac{d^2y}{dz^2} + \frac{Y^2-z^2}{z}y = 0.  \label{uhreq}
%\end{eqnarray}
%Such an equation describes an absorption of an extraordinary electromagnetic wave in the vicinity of an upper-hybrid resonance providing the wave propagates along the electron concentration gradient and an external magnetic field is weak $(Y \ll 1)$. Physics of the process is irrelevant for the purpose of this paper so we allow $Y$ to take any values.
%
%An effective Stokes diagram for Eq.\ref{uhreq} is shown on Fig.\ref{uhr_sd}. To obtain a brunching structure preserving equation, let`s find Frobenius indexes for Eq.\ref{uhreq}. According to Eq.\ref{chareq}, $f_1=0$ and $f_2=1$, and the equation is
%\begin{eqnarray}
%Tr(CS[s_+]W[w]S[s_0]W[-w^*]S[s_-])=2,  
%\end{eqnarray}
%or, using an explicit form of the operators,
%\begin{eqnarray}
%s_0e^{2Re(w)} + s_+e^{2iIm(w)} + s_-e^{-2iIm(w)} +s_-s_0s_+e^{2Re(w)}=2i.   \label{uhr_frob}
%\end{eqnarray}
%
%To reduce the number of independent unknowns we use a conjugation symmetry of Eq.\ref{uhreq} exactly as we did in a case of the Budden problem. A result is analogous and typical for an equation with the real-valued potential: $s_0=-s_0^*$, $s_+=-s_-^*$. Taking into consideration that the reflection and absorption coefficients in this case have the same expressions in terms of the Stokes constants as in the previous example, let`s define three independent real unknowns $p, \rho$ and $\phi$ by $s_0=2ip$ and $s_-=-\rho e^{i\phi}$. In terms of the new variables Eq.\ref{uhr_frob} can be rewritten as
%\begin{eqnarray}
%p (1-\rho^2) e^{2Re(w)} - \rho sin(\phi-2Im(w)) - 1 = 0,   \label{realFrobEq}
%\end{eqnarray}
%where $\rho=|R|$ and $\phi=Arg(R)$. It is simpler than the initial one, but it is still the only one equation with three real unknowns and it cannot be solved exactly.
%
%Let's try to solve it in case of almost full reflection, $1-\rho \ll 1$. In this case we can express the absorption coefficient $A=1-|R|^2=1-\rho^2$ right away:
%\begin{eqnarray}
%A=p^{-1}(1 + sin(\phi-2Im(w)))e^{-2Re(w)}. \label{allRefl}
%\end{eqnarray}
%An assumption $1-\rho \ll 1$ induce a relation $A \ll 1$, and according to Eq.\ref{allRefl} this is true if $Re(w) \gg 1$. But this last inequality allows us to use an approximation of isolated singularities - in a limit of large values of $w$ $p$ approaches 1/2 and $\phi$ approaches $-\pi/2$, and
%\begin{eqnarray}
%A=4sin^2(Im(w))e^{-2Re(w)}.   \label{ais}
%\end{eqnarray}
%The expression shows how the absorption coefficient behaves for large values of $Y$. The most important feature of this result is a footprint of an interference in the system - there are strict zeros of absorption for $Im(w)=\pi n$. This non trivial result could not be obtained in another way - an approximation of isolated singularities ignores the interference because it knows nothing about the actual structure of the potential. A comparison of Eq.\ref{ais} with the numerical result and a simple approximation of isolated singularities is shown on Fig.\ref{uhr_comp}.


\section{Conclusion \label{sec:con}}
The example of the Budden potential shows that use of a series representation of a solution gives an additional equation for the Stokes constants. Since the equation reflects actual branching structure of a general solution it can correctly describe the behaviour of the Stokes constants even in the case of non-isolated singularities. Also, different symmetries of a potential can be used to reduce the number of independent Stokes constants.

Certainly this does not guarantee that the relations discussed above are enough to determine all of the Stokes constants. It is possible to combine the ideas outlined in this paper with other approximation such as the use of constants for isolated singularities or isolated pairs of singularities.

\begin{thebibliography}{30}

\bibitem{frbook} Fr\"oman N. and Fr\"oman P.O. \textit{Physical Problems Solved by the Phase-Integral Method} (Cambridge: Cambridge University Press, 2002)

\bibitem{frpaper} Fr\"oman N., Fr\"oman P.O. and Lundborg B. \textit{The Stokes constants for a cluster of transition
points} Math. Proc. Cambridge Philos. Soc. \textbf{104} (1988), 153-179.

\bibitem{zwaan} Zwaan A., \textit{Intensit\"aten im Ca-Funkenspektrum}, Academish proefschrift thesis (Utrecht, 1929)

\bibitem{stokes} Stokes, G. G., Trans. Camb. Phil. Soc. \textbf{10}, 105 (1857).

\bibitem{white-chen} R. B. White, and F. F. Chen, {Plasma Physics} \textbf{16}(7), 565 (1974).

\bibitem{budden} Budden, K. G., Phil. Trans. Royal Soc. London 290, 405 (1979).

\bibitem{rwbook} R. B. White,
 {\it Asymptotic Analysis of Differential Equations}, Imperial College Press, 2010.

\bibitem{heading} J. Heading. {\it An Introduction to Phase Integral Methods} 
Wiley, NY (1962)

\bibitem{ours} R.B. White, A.Kutlin {\it Bound state energies using Phase integral methods} 

%\bibitem{froman1} Fr\"oman, N., and Fr\"oman, P. O., 1974, Ann Phys (NY) 83, 103–107. (Review: Zentralblatt
%f¨ur Mathematik und ihre Grenzgebiete, Mathematics Abstracts 279, 190–191, 1974.)
%
%\bibitem{froman2} Fr\"oman, N., and Fr\"oman, P. O., 1974, Nuovo Cim 20B, 121–132.

\bibitem{symm} Fr\"oman, N., Fr\"oman, P. O., and Lundborg, B., 1988b, 
Math. Proc. Camb. Phil. Soc. \textbf{104}, 181–191

\bibitem{wkb1} G. Wentzel, Zeit. f. Phys. \textbf{38}, 518 (1926).

\bibitem{wkb2} H. A. Kramers, Zeit. f. Phys. \textbf{39}, 828 (1926).

\bibitem{wkb3} L. Brillion, C. R. Acad. Sci. Paris \textbf{183}, 24 (1926).

\bibitem{wkbj} H. Jeffries, Philos. Mag. [7] 33, 451 (1942)

\bibitem{gamma} M. Abramowitz and I. A. Stegun, eds., 
{\it Handbook of Mathematical Functions With Formulas, Graphs, and Mathematical Tables}, 
NBS Applied Mathematics Series \textbf{55}, National Bureau of Standards, Washington, DC (1964).

\end{thebibliography}
\end{document}